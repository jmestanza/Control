\documentclass{article}

% preambulo:
\input{Control_tp1_preamble.tex}

\begin{document}

\newgeometry{} % margenes default para la caratula
% caratula:
\begin{titlepage}
\newcommand{\HRule}{\rule{\linewidth}{0.5mm}}
\center
\mbox{\textsc{\LARGE \bfseries {Instituto Tecnol\'ogico de Buenos Aires}}}\\[1.5cm]
\textsc{\Large 22.85 - Sistemas de Control}\\[0.5cm]


\HRule \\[0.6cm]
{ \Huge \bfseries Trabajo de Laboratorio N$^{\circ}$1: Phase-Locked Loop (PLL) o Lazo de Enganche de Fase}\\[0.4cm] % Title of your document
\HRule \\[1.5cm]


{\large

\emph{Grupo 1}\\
\vspace{3px}

\begin{tabular}{lr} 	
\textsc{M\'aspero}, Martina  & 57120 \\
\textsc{Mestanza}, Joaqu\'in Mat\'ias  & 58288 \\
\textsc{Nowik}, Ariel Santiago  & 58309 \\
\textsc{Panaggio Venerandi}, Guido Martin  & xxxxx \\
\textsc{Parra}, Roc\'io  & 57669 \\
\textsc{Regueira}, Marcelo Daniel  & 58300 \\

\end{tabular}

\vspace{20px}

\emph{Profesor}\\
\vspace{3px}
\textsc{Nasini}, V\'ictor Gustavo\\ 	
\vspace{100px}

\begin{tabular}{ll}

Presentado: & xx/09/2019\\

\end{tabular}

}

\vfill

\end{titlepage}

% cambio los margenes para el resto del documento
\newgeometry{left=2.5cm, top=2.5cm, right=2cm, bottom=2cm}

% indice:
\tableofcontents
\newpage

\section{Transferencia del sistema a lazo abierto}

En el circuito que simula el sistema físico, se identifican bloques amplificadores inversores con operacionales. Cuatro de ellos son de ganancia -1 y los otros dos (que definirán las variables de estado) funcionan como integradores. Es decir, su transferencia es del formato:

\[
H(S) = -\frac{1}{SCR}
\]

Donde en cada caso, para el primer y segundo integrador respectivamente se obtiene:

\[
H_1(S) = -\frac{10}{S} \hspace{2cm} H_2(S) = -\frac{1000}{47 \cdot S}
\]



\end{document}
