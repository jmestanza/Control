\documentclass{article}

% preambulo:
\input{Control_tp2_preamble.tex}

\begin{document}

\newgeometry{} % margenes default para la caratula
% caratula:
\input{Control_tp2_caratula.tex}

% cambio los margenes para el resto del documento
\newgeometry{left=2.5cm, top=2.5cm, right=2cm, bottom=2cm}

% indice:
\tableofcontents
\newpage

\section{Transferencia del sistema a lazo abierto}

En el circuito que simula el sistema físico, se identifican bloques amplificadores inversores con operacionales. Cuatro de ellos son de ganancia -1 y los otros dos (que definirán las variables de estado) funcionan como integradores. Es decir, su transferencia es del formato:

\[
H(S) = -\frac{1}{SCR}
\]

Donde en cada caso, para el primer y segundo integrador respectivamente se obtiene:

\[
H_1(S) = -\frac{10}{S} \hspace{2cm} H_2(S) = -\frac{1000}{47 \cdot S}
\]

Finalmente, el diagrama en bloques a lazo abierto queda:

\begin{figure}[H]
\centering
\includegraphics[width=0.7\linewidth]{Imagenes/HLazoAbierto.png}
\caption{Transferencia del sistema a lazo abierto}
\label{fig:Circuito}
\end{figure}

Siendo $X_1$ y $X_2$ las variables de estado. Planteando las transferencias intermedias se obtienen las ecuaciones de estado:
\[
\frac{X_2}{U} = \frac{10}{S} \Longrightarrow \dot{x_2} = 10u
\]

\[
\frac{X_1}{X_2} = \frac{1000}{47 \cdot S} \Longrightarrow \dot{x_1} = \frac{1000}{47} \cdot x_2
\]
La ecuación de salida:

\[
y = x_1
\]

Armando el espacio de estados matricial se tiene:

\[
\begin{bmatrix}
\dot{x_1} \\
\dot{x_2} 
\end{bmatrix}
=
\begin{bmatrix}
0 & \frac{1000}{47} \\
0 & 0 
\end{bmatrix}
\cdot
\begin{bmatrix}
x_1 \\
x_2 
\end{bmatrix}
+
\begin{bmatrix}
0 \\
10 
\end{bmatrix}
\cdot
\begin{bmatrix}
u
\end{bmatrix}
\]

\[
\begin{bmatrix}
y
\end{bmatrix}
=
\begin{bmatrix}
1 & 0 
\end{bmatrix}
\cdot
\begin{bmatrix}
x_1 \\
x_2 
\end{bmatrix}
\]

\end{document}
