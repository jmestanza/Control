 \documentclass{article}

% preambulo:
\input{Control_tp4_preamble.tex}

\usepackage{fancyhdr}

\geometry{top=2.5cm, bottom=2.0cm, left=2.25cm, right=2.25cm}

\lhead{Sistemas de Control 22.85}
\chead{TP4 - Control Carrito}
\rhead{ITBA}
\renewcommand{\headrulewidth}{1pt}
\renewcommand{\footrulewidth}{1pt}
\pagestyle{fancy}


\begin{document}

%\newgeometry{} % margenes default para la caratula
% caratula:
\input{Control_tp4_caratula.tex}


% indice:
\tableofcontents
\newpage

\section{PID: Introducción teórica}
Los controladores PID (proporcional, integrador y derivativo) proveen un control de lazo empleando feedback que es utilizado en la industria del control.
Un controlador PID calcula el error e(t) como la diferencia entre el setpoint(deseada) y una variable medida del proceso (la salida de la planta).


\begin{figure}[H]
\centering
\includegraphics[width=0.5\linewidth]{images/PID.jpg}
\caption{Controlador PID: Esquema}
\label{fig:PID}
\end{figure}

Como se puede observar en la figura \ref{fig:PID} en los controladores PID se dispone de 3 constantes.
\begin{itemize}
  \item $K_p$: constante que acompaña al error 
  \item $K_i$: constante que acompaña a la integral del error 
  \item $K_d$: constante que acompaña a la derivada del error
  
\end{itemize}
\newpage
\section{Práctica}
Se ajustaron las constantes mediante el siguiente método:
\begin{itemize}
  \item Primero establecer $K_i=0$ y $K_d=0$. 
  \item Incrementar la $K_p$ hasta que la salida oscile
  \item Establecer $K_p$ a aproximadamente la mitad del valor configurado previamente
  \item Incrementar $K_i$ hasta que el proceso se ajuste en el tiempo requerido (precaución: subir mucho I puede causar inestabilidad)
  \item Finalmente, incrementar D si se necesita hasta que el lazo sea lo suficientemente rápido para alcanzar su referencia tras una variación brusca de la carga.
 
\newpage
\section{Resultados}
 
  
\end{itemize}
 

\end{document}
